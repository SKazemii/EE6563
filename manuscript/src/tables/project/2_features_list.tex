{\small
\hspace*{-4cm}\begin{tabularx}{\linewidth}{@{}rlcc@{}}
\caption{list of features} \\
\label{tab:Features_list}\\

\toprule
  \#  &  Features Name & {Categories Name } & \# features \\
\midrule
\endfirsthead
\toprule
  \#  &  Features Name & {Categories Name } & \# features \\
\midrule
\endhead
\midrule
\multicolumn{4}{r}{\footnotesize( To be continued)}
\endfoot
\bottomrule
\endlastfoot

%  1 & ECDF              & Statistical  & A\\
%  2 & ECDF Percentile      & Statistical  & A         \\
%  3 & ECDF Percentile Count & Statistical  & A\\
  1 & Histogram    & Statistical  & 10 \\
%  5 & Interquartile range               & Statistical  & A \\
%  6 & Kurtosis                   & Statistical  & A \\
  2 & Max         & Statistical  & 1 \\
  3 & Mean         & Statistical   & 1\\
  4 & Mean absolute deviation         & Statistical   & 1\\
  5 & Median         & Statistical   & 1\\
  6 & Median abs deviation         & Statistical   & 1\\
  7 & Min         & Statistical   & 1\\
  8 & Root mean square         & Statistical   & 1\\
  %9 & Skewness         & Statistical   & A\\
  9 & Standard deviation         & Statistical   & 1\\
  10 & Variance         & Statistical   & 1 \\ \hline
   11 & AR coefficients & \Gls{AR} & 8\\\hline
  12 & Absolute energy         & Temporal  & 1 \\%& The absolute energy of the signal.\\
  13 & Area under the curve         & Temporal   & 1\\ %& Computes the area under the curve of the signal.\\
  %19 & Autocorrelation         & Temporal   & -1\\
  14 & Centroid         & Temporal   & 1 \\%& Computes the centroid along the time axis\\
  15 & Entropy         & Temporal   & 1 \\%& Computes the Shannon entropy of the signal\\
  16 & Mean absolute diff         & Temporal   & 1\\ %& Computes mean absolute differences of the signal\\
  17 & Mean diff         & Temporal   & 1 \\%& Computes mean of differences of the signal.\\
  18 & Median absolute diff         & Temporal   & 1 \\%& Computes median absolute differences of the signal.\\
  19 & Median diff         & Temporal   & 1 \\%& Computes median of differences of the signal. \\
  20 & Negative turning points         & Temporal   & 1\\ %& Computes number of negative turning points of the signal.\\
  %28 & Neighbourhood peaks         & Temporal   & 1 & Computes the number of peaks from a neighbourhood of the signal.\\
  21 & Positive turning points         & Temporal   & 1 \\%& Computes number of positive turning points of the signal. \\
  %30 & Signal distance         & Temporal   & -1 & \\
  22 & Slope         & Temporal   & 1 \\\hline%& by fitting a linear equation to the observed data.\\
  %32 & Sum absolute diff         & Temporal   & A\\
  %33 & Total energy         & Temporal   & A\\
  %34 & Zero crossing rate         & Temporal   & -1\\
  %35 & Neighbourhood peaks         & Temporal   & A\\
  23 & FFT mean coefficient         & Spectral   & 256\\
  %37 & Fundamental frequency         & Spectral   & A\\
  %38 & Human range energy         & Spectral   & A\\
  %39 & LPCC         & Spectral   & A\\
  %40 & MFCC         & Spectral   & A\\
  24 & Max power spectrum         & Spectral   & 1\\
  25 & Maximum frequency         & Spectral   & 1\\
  26 & Median frequency         & Spectral   & 1\\
  %44 & Power bandwidth         & Spectral   & A\\
  27 & Spectral centroid         & Spectral   & 1\\
  %46 & Spectral decrease         & Spectral   & A\\
  %47 & Spectral distance         & Spectral   & A\\
  28 & Spectral entropy         & Spectral   & 1\\
  %49 & Spectral kurtosis         & Spectral   & A\\
  %50 & Spectral turning points         & Spectral   & A\\
  %51 & Spectral roll-off         & Spectral   & A\\
  %52 & Spectral roll-on         & Spectral   & A\\
  %53 & Spectral skewness         & Spectral   & A\\
  %54 & Spectral slope         & Spectral   & A\\
  %55 & Spectral spread         & Spectral   & A\\
  %56 & Spectral variation         & Spectral   & A\\
  29 & Wavelet abs mean         & Spectral   & 10\\
  30 & Wavelet energy         & Spectral   & 10\\
  31 & Wavelet stand deviation         & Spectral   & 10\\
  32 & Wavelet entropy         & Spectral   & 1\\
  33 & Wavelet variance         & Spectral   & 10\\ \hline
  34 & Inter stride         & Temporal   & 1\\ \hline
\multicolumn{3}{r}{The total number of features} & 341\\
\end{tabularx}\hspace*{-4cm}
}


