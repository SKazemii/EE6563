\documentclass[12pt]{article}

\usepackage{preamble}

%! Author = sbbfti
%! Date = 10/06/2020

\newacronym{ADF}{ADF test}{Augmented Dickey-Fuller test}
\newacronym{KPSS}{KPSS test}{Kwiatkowski-Phillips-Schmidt-Shin test}
\newacronym{ACF}{ACF}{AutoCorrelation function}
\newacronym{PACF}{PACF}{Partial AutoCorrelation function}


\newacronym{ti}{$T_{i}$}{indoor air temperature, $^{\circ}$C}



\begin{document}

%%%%%%%%%%%%%%%%%%%%%%%%%%%%%%%%%%%%%%%%%%%%%%%%%%%%%%%%%%%%%%%%%


\begin{titlepage}
	\centering
    \vspace*{0.4 cm}
    \includegraphics[scale = 0.5]{figures/unb.jpg}\\[1.0 cm]	% University Logo
    \textsc{\LARGE \newline\newline University of New Brunswick}\\[1.8 cm]	% University Name
	\textsc{\Large Time Series Analysis\\(EE 6563)}\\[0.5 cm]				% Course Code
	\rule{\linewidth}{0.2 mm} \\[0.4 cm]
	{ \huge \bfseries \thetitle}\\
	\rule{\linewidth}{0.2 mm} \\[1.5 cm]
	
	\begin{minipage}{0.5\textwidth}
		\begin{flushleft} \large
			\emph{Professor:}\\
			Erik Scheme\\
            Electrical and Computer Engineering\\
			\end{flushleft}
			\end{minipage}~
			\begin{minipage}{0.5\textwidth}
            
			\begin{flushright} \large
			\emph{Author:} \\
			Saeed Kazemi\\ (3713280)\\

		\end{flushright}
        
	\end{minipage}\\[1 cm]
	
	
    \thedate
    
    
    
	
\end{titlepage}

%%%%%%%%%%%%%%%%%%%%%%%%%%%%%%%%%%%%%%%%%%%%%%%%%%%%%%%%%%%%%%%%%



%\tableofcontents
\pagebreak

%%%%%%%%%%%%%%%%%%%%%%%%%%%%%%%%%%%%%%%%%%%%%%%%%%%%%%%%%%%%%%%%%
%================================================================
\begin{enumerate}


%%%%%%%%%%%%%%%%%%%%%%%%%%%%%%%%%%%%%%%%%%%%%%%%%%%%%%%%%%%%%%%%%
%%%%%%%%%%%%%%%%%%%%%%%% Question 1 %%%%%%%%%%%%%%%%%%%%%%%%%%%%%
%%%%%%%%%%%%%%%%%%%%%%%%%%%%%%%%%%%%%%%%%%%%%%%%%%%%%%%%%%%%%%%%%
\item \textbf{For each dataset, visualize the raw signals, identify any trends, seasonality, and/or other components, and try to remove them. Remember that you are not limited to the tools shown in the tutorial and should explore the various concepts discussed in class. You do not need to explain the theory behind the approaches, but you should provide justification for their use, and discussion of their results.}


\textit{Figure \ref{fig:Ass1_D1_raw_signal} and \ref{fig:Ass1_D2_raw_signal} indicate the raw signal of both data sets. Furthermore, figure \ref{fig:Ass1_D1_raw_signal_1986} to  \ref{fig:Ass1_D2_raw_signal_1990} illustrate a period of two datasets. }

\begin{figure}[H]
    \centering
    \begin{minipage}[b]{1\textwidth}
        \includegraphics[width=\textwidth]{figures/Ass1/Ass1_D1_raw_signal.png}
    \end{minipage}
    \caption{Visualizing the raw signals of the first dataset.}
    \label{fig:Ass1_D1_raw_signal}
\end{figure}

\begin{figure}[H]
    \centering
    \begin{minipage}[b]{1\textwidth}
        \includegraphics[width=\textwidth]{figures/Ass1/Ass1_D2_raw_signal.png}
    \end{minipage}
    \caption{Visualizing the raw signals of the second dataset.}
    \label{fig:Ass1_D2_raw_signal}
\end{figure}

\begin{figure}[H]
    \centering
    \begin{minipage}[b]{1\textwidth}
        \includegraphics[width=\textwidth]{figures/Ass1/Ass1_D1_raw_signal_1986.png}
    \end{minipage}
    \caption{Visualizing the raw signals of the first dataset in 1986.}
    \label{fig:Ass1_D1_raw_signal_1986}
\end{figure}

\begin{figure}[H]
    \centering
    \begin{minipage}[b]{1\textwidth}
        \includegraphics[width=\textwidth]{figures/Ass1/Ass1_D1_raw_signal_1990.png}
    \end{minipage}
    \caption{Visualizing the raw signals of the first dataset in 1990.}
    \label{fig:Ass1_D1_raw_signal_1990}
\end{figure}

\begin{figure}[H]
    \centering
    \begin{minipage}[b]{1\textwidth}
        \includegraphics[width=\textwidth]{figures/Ass1/Ass1_D2_raw_signal_1990.png}
    \end{minipage}
    \caption{Visualizing the raw signals of the second dataset in 1990.}
    \label{fig:Ass1_D2_raw_signal_1990}
\end{figure}

\textit{For intuition, Table \ref{tab:Ass1_D1_raw_signal} to  \ref{tab:Ass1_D2_raw_signal_summary_statistics} show the top raw of two datasets along with the summary of two time series datasets.}


\begin{table}[H]
 \centering
\caption{The top row of the raw signal of the first dataset (daily temp).
{\label{tab:Ass1_D1_raw_signal}}}
\input{tables/Ass1/Ass1_D1_raw_signal.tex}
\end{table}

\begin{table}[H]
 \centering
\caption{The description of the first dataset.
{\label{tab:Ass1_D1_raw_signal_summary_statistics}}}
\input{tables/Ass1/Ass1_D1_raw_signal_summary_statistics.tex}
\end{table}

\begin{table}[H]
 \centering
\caption{The top row of the raw signal of the second dataset.
{\label{tab:Ass1_D2_raw_signal}}}
\input{tables/Ass1/Ass1_D2_raw_signal.tex}
\end{table}

\begin{table}[H]
 \centering
\caption{The description of the second dataset.
{\label{tab:Ass1_D2_raw_signal_summary_statistics}}}
\input{tables/Ass1/Ass1_D2_raw_signal_summary_statistics.tex}
\end{table}






\textit{For decomposing the data the below methods were used:}
    \begin{enumerate}
    \item \textit{Seasonal\_decompose (Figure
        \ref{fig:Ass1_D1_seasonal_decompose} and \ref{fig:Ass1_D2_seasonal_decompose})}
        
    \item \textit{STL (Figure
        \ref{fig:Ass1_D1_STL} and \ref{fig:Ass1_D2_STL})}
        
    \item \textit{Linear Regression method (Figure
        \ref{fig:Ass1_D1_LinearRegression_diff} and \ref{fig:Ass1_D2_LinearRegression_diff})}
        
    \item \textit{Difference method (Figure
        \ref{fig:Ass1_D1_one_diff} and \ref{fig:Ass1_D2_one_diff})}
        
    \item \textit{Fitting a polynomial (Figure
        \ref{fig:Ass1_D1_fiting_polynomial} and \ref{fig:Ass1_D2_fiting_polynomial})}
        
    \item \textit{Moving Average window (Figure
        \ref{fig:Ass1_D1_Moving_Avrage} and \ref{fig:Ass1_D2_Moving_Avrage})}

    \end{enumerate}
    
\textit{Some mentioned methods are used only for extracting only one component while others like STL and Seasonal\_decompose provided all three components. Table \ref{tab:Ass1_comparing_methods} compares these} methods together.

\textit{Also there are two model for reconstruction of time series, Additive Model and Multiplicative Model. In this assignment the additive model was used because Multiplicative is not appropriate for zero and negative values.}

\begin{table}[H]
\centering
\caption{Comparing the implemented methods.
\label{tab:Ass1_comparing_methods}}
\begin{tabular}{@{}lccc@{}}
\toprule
                    & \begin{tabular}[c]{@{}c@{}}Trend \\ component\end{tabular} & \begin{tabular}[c]{@{}c@{}}Seasonal \\ component\end{tabular} & \begin{tabular}[c]{@{}c@{}}Residual \\ component\end{tabular} \\ \midrule
Seasonal\_decompose & \checkmark                                  & \checkmark                                     & \checkmark                                     \\ \midrule
STL                 & \checkmark                                  & \checkmark                                     & \checkmark                                     \\ \midrule
Linear Regression   & \checkmark                                  & -                                                             & -                                                             \\ \midrule
Difference          & -                                                          & \checkmark                                     & -                                                             \\ \midrule
Moving average      & \checkmark                                  & -                                                             & -                                                             \\ \bottomrule
\end{tabular}

\end{table}




\begin{figure}[H]
    \centering
    \begin{minipage}[b]{1\textwidth}
        \includegraphics[width=\textwidth]{figures/Ass1/Ass1_D1_seasonal_decompose.png}
    \end{minipage}
    \caption{Decomposition of the first dataset by seasonal\_decompose method}
    \label{fig:Ass1_D1_seasonal_decompose}
\end{figure}

\begin{figure}[H]
    \centering
    \begin{minipage}[b]{1\textwidth}
        \includegraphics[width=\textwidth]{figures/Ass1/Ass1_D2_seasonal_decompose.png}
    \end{minipage}
    \caption{Decomposition of the second dataset by seasonal\_decompose method}
    \label{fig:Ass1_D2_seasonal_decompose}
\end{figure}

\begin{figure}[H]
    \centering
    \begin{minipage}[b]{1\textwidth}
        \includegraphics[width=\textwidth]{figures/Ass1/Ass1_D1_STL.png}
    \end{minipage}
    \caption{Decomposition of the first dataset by STL method}
    \label{fig:Ass1_D1_STL}
\end{figure}

\begin{figure}[H]
    \centering
    \begin{minipage}[b]{1\textwidth}
        \includegraphics[width=\textwidth]{figures/Ass1/Ass1_D2_STL.png}
    \end{minipage}
    \caption{Decomposition of the second dataset by STL method}
    \label{fig:Ass1_D2_STL}
\end{figure}

\begin{figure}[H]
    \centering
    \begin{minipage}[b]{1\textwidth}
        \includegraphics[width=\textwidth]{figures/Ass1/Ass1_D1_LinearRegression_diff.png}
    \end{minipage}
    \caption{Decomposition of the first dataset by LinearRegression and difference method.}
    \label{fig:Ass1_D1_LinearRegression_diff}
\end{figure}

\begin{figure}[H]
    \centering
    \begin{minipage}[b]{1\textwidth}
        \includegraphics[width=\textwidth]{figures/Ass1/Ass1_D2_LinearRegression_diff.png}
    \end{minipage}
    \caption{Decomposition of the second dataset by LinearRegression and difference method.}
    \label{fig:Ass1_D2_LinearRegression_diff}
\end{figure}

\begin{figure}[H]
    \centering
    \begin{minipage}[b]{1\textwidth}
        \includegraphics[width=\textwidth]{figures/Ass1/Ass1_D1_one_diff.png}
    \end{minipage}
    \caption{Detrending of the first dataset by difference method.}
    \label{fig:Ass1_D1_one_diff}
\end{figure}

\begin{figure}[H]
    \centering
    \begin{minipage}[b]{1\textwidth}
        \includegraphics[width=\textwidth]{figures/Ass1/Ass1_D2_one_diff.png}
    \end{minipage}
    \caption{Detrending of the second dataset by difference method.}
    \label{fig:Ass1_D2_one_diff}
\end{figure}

\begin{figure}[H]
    \centering
    \begin{minipage}[b]{1\textwidth}
        \includegraphics[width=\textwidth]{figures/Ass1/Ass1_D1_fiting_polynomial.png}
    \end{minipage}
    \caption{Seasonal component of the first dataset by fitting a polynomial.}
    \label{fig:Ass1_D1_fiting_polynomial}
\end{figure}

\begin{figure}[H]
    \centering
    \begin{minipage}[b]{1\textwidth}
        \includegraphics[width=\textwidth]{figures/Ass1/Ass1_D2_fiting_polynomial.png}
    \end{minipage}
    \caption{Seasonal component of the second dataset by fitting a polynomial.}
    \label{fig:Ass1_D2_fiting_polynomial}
\end{figure}

\begin{figure}[H]
    \centering
    \begin{minipage}[b]{1\textwidth}
        \includegraphics[width=\textwidth]{figures/Ass1/Ass1_D1_Moving_Avrage.png}
    \end{minipage}
    \caption{Seasonal component of the first dataset by Moving Average.}
    \label{fig:Ass1_D1_Moving_Avrage}
\end{figure}

\begin{figure}[H]
    \centering
    \begin{minipage}[b]{1\textwidth}
        \includegraphics[width=\textwidth]{figures/Ass1/Ass1_D2_Moving_Avrage.png}
    \end{minipage}
    \caption{Seasonal component of the second dataset by Moving Average.}
    \label{fig:Ass1_D2_Moving_Avrage}
\end{figure}




%%%%%%%%%%%%%%%%%%%%%%%%%%%%%%%%%%%%%%%%%%%%%%%%%%%%%%%%%%%%%%%%%
%%%%%%%%%%%%%%%%%%%%%%%% Question 2 %%%%%%%%%%%%%%%%%%%%%%%%%%%%%
%%%%%%%%%%%%%%%%%%%%%%%%%%%%%%%%%%%%%%%%%%%%%%%%%%%%%%%%%%%%%%%%%
\item \textbf{For each dataset, examine the stationarity of the residuals using the ACF and PACF functions, Lag Plots, and/or other approaches. Show your results and provide commentary about your observations.}

\textit{\gls{ADF} shows that the time series is a non-stationary or stationary. If the p-value is very less than significance level of 0.05 and the \gls{ADF} statistic is lower than one of the critical values, then the time series is a stationary. For example, table \ref{tab:Ass1_D1_ADF} shows a stationary time series.}

\begin{table}[H]
\centering
\caption{The result of the \gls{ADF} on the first dataset.}
\label{tab:Ass1_D1_ADF}
\begin{tabular}{lr}
\toprule
{} &            0 \\
\midrule
ADF Statistic               &   -19.036532 \\
p-value                     &     0.000000 \\
\#Lags Used                  &     5.000000 \\
Number of Observations Used &  3279.000000 \\
Critical Value (1\%)         &    -3.432346 \\
Critical Value (5\%)         &    -2.862422 \\
Critical Value (10\%)        &    -2.567239 \\
\bottomrule
\end{tabular}

\end{table}

\begin{table}[H]
\centering
\caption{The result of the \gls{ADF} on the second dataset.}
\label{tab:Ass1_D2_ADF}
\begin{tabular}{lr}
\toprule
{} &             0 \\
\midrule
ADF Statistic               & -1.416763e+01 \\
p-value                     &  2.029256e-26 \\
\#Lags Used                  &  2.400000e+01 \\
Number of Observations Used &  2.795000e+03 \\
Critical Value (1\%)         & -3.432692e+00 \\
Critical Value (5\%)         & -2.862575e+00 \\
Critical Value (10\%)        & -2.567321e+00 \\
\bottomrule
\end{tabular}

\end{table}


\textit{ \gls{KPSS} is another test for checking the stationarity of a time series. If the p-value is very less than significance level of 0.05, then the time series is not a stationary. Table \ref{tab:Ass1_D1_KPSS} and \ref{tab:Ass1_D2_KPSS} show the result of  this test.}

\begin{table}[H]
\centering
\caption{The result of the \gls{KPSS} on the first dataset.}
\label{tab:Ass1_D1_KPSS}
\input{tables/Ass1/Ass1_D1_KPSS.tex}
\end{table}

\begin{table}[H]
\centering
\caption{The result of the \gls{KPSS} on the second dataset.}
\label{tab:Ass1_D2_KPSS}
\begin{tabular}{lr}
\toprule
{} &          0 \\
\midrule
KPSS Statistic        &   0.017052 \\
p-value               &   0.100000 \\
Lags Used             &  68.000000 \\
Critical Value (10\%)  &   0.347000 \\
Critical Value (5\%)   &   0.463000 \\
Critical Value (2.5\%) &   0.574000 \\
Critical Value (1\%)   &   0.739000 \\
\bottomrule
\end{tabular}

\end{table}

\textit{We can conclude that the series is stationary or not based on the both result of \gls{KPSS} and \gls{ADF} \cite{StationarityStatsmodels}. Table \ref{tab:1} shows possible outcomes of applying these two tests.}

\begin{table}[H]
\centering
\caption{The combination of the result of the \gls{KPSS} and \gls{ADF}.}
\label{tab:1}
\input{tables/Ass1/1.tex}
\end{table}



\textit{\gls{ACF}  and \gls{PACF} plots allow you to determine the time series at zero hoe much correlation has with other lags. Figure \ref{fig:Ass1_D1_PACF_ACF} and \ref{fig:Ass1_D2_PACF_ACF} indicate these two plot for our datasets.}



\begin{figure}[H]
    \centering
    \begin{minipage}[b]{1\textwidth}
        \includegraphics[width=\textwidth]{figures/Ass1/Ass1_D1_PACF_ACF.png}
    \end{minipage}
    \caption{A plot of the \gls{ACF} and \gls{PACF} of the first dataset.}
    \label{fig:Ass1_D1_PACF_ACF}
\end{figure}

\begin{figure}[H]
    \centering
    \begin{minipage}[b]{1\textwidth}
        \includegraphics[width=\textwidth]{figures/Ass1/Ass1_D2_PACF_ACF.png}
    \end{minipage}
    \caption{A plot of the \gls{ACF} and \gls{PACF} of the second dataset.}
    \label{fig:Ass1_D2_PACF_ACF}
\end{figure}


\textit{These two plots are used to find the q and p for the ARIMA model. For example, if \gls{ACF} decays towards zero, and \gls{PACF} have only q significant value then our time series is a AR(q) process.}

\begin{table}[H]
\centering
\caption{The combination of the result of the \gls{KPSS} and \gls{ADF}.}
\label{tab:1}
\input{tables/Ass1/1.tex}
\end{table}

\textit{The below figures (figure \ref{fig:Ass1_D1_Lag_Plots} and \ref{fig:Ass1_D2_Lag_Plots} ) show the lag plot of the two datasets. In each figure, there are four lag plots. As these plots illustrate, both datasets are linear, therefore our time series are a AR process.  }


\begin{figure}[H]
    \centering
    \begin{minipage}[b]{1\textwidth}
        \includegraphics[width=\textwidth]{figures/Ass1/Ass1_D1_Lag_Plots.png}
    \end{minipage}
    \caption{A different lag plot of the first dataset.}
    \label{fig:Ass1_D1_Lag_Plots}
\end{figure}

\begin{figure}[H]
    \centering
    \begin{minipage}[b]{1\textwidth}
        \includegraphics[width=\textwidth]{figures/Ass1/Ass1_D2_Lag_Plots.png}
    \end{minipage}
    \caption{A different lag plot of the second dataset.}
    \label{fig:Ass1_D2_Lag_Plots}
\end{figure}




\begin{figure}[H]
    \centering
    \begin{minipage}[b]{1\textwidth}
        \includegraphics[width=\textwidth]{figures/Ass1/Ass1_D1_Lag_Plots_residual.png}
    \end{minipage}
    \caption{A different lag plot of residual component of the first dataset.}
    \label{fig:Ass1_D1_Lag_Plots_residual}
\end{figure}

\begin{figure}[H]
    \centering
    \begin{minipage}[b]{1\textwidth}
        \includegraphics[width=\textwidth]{figures/Ass1/Ass1_D2_Lag_Plots_residual.png}
    \end{minipage}
    \caption{A different lag plot of residual component of the second dataset.}
    \label{fig:Ass1_D2_Lag_Plots_residual}
\end{figure}










%%%%%%%%%%%%%%%%%%%%%%%%%%%%%%%%%%%%%%%%%%%%%%%%%%%%%%%%%%%%%%%%%
%%%%%%%%%%%%%%%%%%%%%%%% Question 3 %%%%%%%%%%%%%%%%%%%%%%%%%%%%%
%%%%%%%%%%%%%%%%%%%%%%%%%%%%%%%%%%%%%%%%%%%%%%%%%%%%%%%%%%%%%%%%%
\item \textbf{Try modeling the residuals as an AR process. Use the tools at your disposal to decide on an appropriate order and analyse the results. What is the impact of selecting different orders on the remaining residuals?}





%so these are my starting points
%must be stationary
%these two plot do not tell us much
\textit{For this part, we need to select the order of the \gls{AR} term (p). Therefore, \gls{PACF} and \gls{ACF} were plotted at first step (see figures \ref{fig:Ass1_D1_PACF_ACF_X} and \ref{fig:Ass1_D2_PACF_ACF_X}).}  

\begin{figure}[H]
    \centering
    \begin{minipage}[b]{1\textwidth}
        \includegraphics[width=\textwidth]{figures/Ass1/Ass1_D1_PACF_ACF_X.png}
    \end{minipage}
    \caption{A plot of the \gls{PACF} and \gls{ACF} of the residual part of the first dataset (residual of the STL method).}
    \label{fig:Ass1_D1_PACF_ACF_X}
\end{figure}

\begin{figure}[H]
    \centering
    \begin{minipage}[b]{1\textwidth}
        \includegraphics[width=\textwidth]{figures/Ass1/Ass1_D2_PACF_ACF_X.png}
    \end{minipage}
    \caption{A plot of the \gls{PACF} and \gls{ACF} of the residual part of the second dataset (residual of the STL method).}
    \label{fig:Ass1_D2_PACF_ACF_X}
\end{figure}


\textit{Sinse \gls{ACF} decaying in both figures, it can be concluded that the process is an Auto Regressive process. Also, based on \gls{PACF} in the first plot (figure \ref{fig:Ass1_D1_PACF_ACF_X}), the parameter of the \gls{AR} model should be start with lags 1 and 2 due to these two lags have a significant value. Likewise, for the second dataset, we should start with lags 1 to lags 4 (see figure \ref{fig:Ass1_D2_PACF_ACF_X}).}

\begin{figure}[H]
    \centering
    \begin{minipage}[b]{1\textwidth}
        \includegraphics[width=\textwidth]{figures/Ass1/Ass1_D1_ARs models.png}
    \end{minipage}
    \caption{The residual signal (blue) and the prediction of \gls{AR} (orange) for the first dataset.}
    \label{fig:Ass1_D1_ARs_models}
\end{figure}

\begin{figure}[H]
    \centering
    \begin{minipage}[b]{1\textwidth}
        \includegraphics[width=\textwidth]{figures/Ass1/Ass1_D2_ARs models.png}
    \end{minipage}
    \caption{The residual signal (blue) and the prediction of \gls{AR} (orange) for the second dataset.}
    \label{fig:Ass1_D2_ARs_models}
\end{figure}


\textit{Figure \ref{fig:Ass1_D1_ARs_models} and \ref{fig:Ass1_D2_ARs_models} indicate the output of our models on the datasets. In addition, tables \ref{tab:Ass1_D1_AR} and \ref{tab:Ass1_D2_AR} show others parameters of these models. }

\begin{table}[H]
\centering
\caption{Comparing the \gls{AR} models in the first dataset.}
\label{tab:Ass1_D1_AR}
\begin{tabular}{lll}
\toprule
{} &  AR(1) &  AR(2) \\
\midrule
p  &      1 &      2 \\
AIC       &  1.288 &  1.277 \\
BIC       &  1.293 &  1.284 \\
RMS error &  1.149 &  1.156 \\
\bottomrule
\end{tabular}

\end{table}

\begin{table}[H]
\centering
\caption{Comparing the \gls{AR} models in the second dataset.}
\label{tab:Ass1_D2_AR}
\begin{tabular}{lllll}
\toprule
{} &  AR(1) &  AR(2) &   AR(3) &   AR(4) \\
\midrule
p  &      1 &      2 &       3 &       4 \\
AIC       &  5.134 &  5.090 &   5.055 &   5.040 \\
BIC       &  5.140 &  5.098 &   5.065 &   5.053 \\
RMS error &  7.780 &  8.643 &  10.094 &  11.064 \\
\bottomrule
\end{tabular}

\end{table}







\textit{\Gls{AIC} and \gls{BIC} that show simplicity and goodness of a \gls{AR} model. The model that has a lower \gls{AIC} and \gls{BIC} is generally better than others for example, AR() .}




%%%%%%%%%%%%%%%%%%%%%%%%%%%%%%%%%%%%%%%%%%%%%%%%%%%%%%%%%%%%%%%%%
%%%%%%%%%%%%%%%%%%%%%%%% Question 4 %%%%%%%%%%%%%%%%%%%%%%%%%%%%%
%%%%%%%%%%%%%%%%%%%%%%%%%%%%%%%%%%%%%%%%%%%%%%%%%%%%%%%%%%%%%%%%%
\input{Questions/Ass1-Q4}








\end{enumerate}



\newpage
\bibliography{references}


\newpage
\section{Appendix (codes)}
\subsection{The script of Lab-1}

\begin{lstlisting}
%% Lab 1: Discrete-Time Frequency
% 
% Author: Maryhelen Stevenson
%
%% Explanation of the function ct_dt(.)
%
% The file ct_dt.m was supplied for use in this lab; a listing of the file
% is included in Appendix 1. The file defines a matlab function
% ct_dt(A,f0,PHI,fs,nc,ifig) which can be used to plot nc cycles
% of a continuous-time cosine with amplitude A, frequency f0, and phase
% PHI.  It also superimposes the values of the discrete-time sinusoid that
% would result from sampling the continuous-time sinusoid at a rate of fs
% samples per second.  The function returns a vector of time
% instances at which the values of the continuous-time sinusoid were
% evaluated to generate the continuous-time plot.
% 
% 
% An example to illustrate the usage of ct_dt(.) follows: 
%
close all
A = 2; % amplitude of continuous-time cosine
f0 = 0.5; % frequency (in units of Hz.) of continuous-time cosine
PHI = -pi/4; % phase (in units of radians) of continuous-time cosine
fs = 10; % sampling rate to be used (units of samples/second)
nc = 5; % number of cycles of continuous-time cosine to be plotted
ifig = 1; % optional figure number to use in the title of the figure
% the function returns the time vector used to plot the c.t. signal
t = ct_dt(A,f0,PHI,fs,nc,ifig); 

%%
% _Discussion of Figure 1_
% In accordance with the usage of ct_dt, we note that Figure 1,
% contains 5 cycles of the continuous-time cosine, $$x_a(t)$,
% where $$x_a(t)=2\cos(2\pi(0.5)t - \pi/4)$.  It also superimposes the 
% discrete-time sinusoid x[n] = xa(n/10).  The values of n are not shown
% but could be added in by hand.  Note that t=0 corresponds to n=0; whereas
% t=1 corresponds to n=10.
% etc.
% 
%% Exercise 1
%  
% Let x_a(t) = 3 sin(2 pi 50 t) = 3 cos (2 pi 50 t - pi/2)
% 
% Define x[n] = x_a(n/fs)
%
% a) Figure 2 shows a plot of xa(t) and x[n] for the case when fs=200
% samples/second.  It was produced using the code below.
% 
%  include the necessary code
close all
A = 3; 
f0 = 50; 
PHI = -pi/2; 
fs = 200; 
nc = 6;
ifig = 2; 


t = ct_dt(A,f0,PHI,fs,nc,ifig); 


%%
% _Discussion of Figure 2_
%
% include discussion here.  Your discussion should include answers to all
% questions posed in the lab manual.  Please use complete sentences.  Keep
% in mind that a reader should not have to have a copy of the lab manual to
% make sense of your discussion.
%%
% b) Figure 3 shows a plot of xa(t) and x[n] for the case when fs=120
% samples/second. It was produced using the code below.
% 
%  include the necessary code
fs = 120; 
ifig = 3; 


t = ct_dt(A,f0,PHI,fs,nc,ifig); 
%%
% _Discussion of Figure 3_
%
% include discussion here
%%
% c) Figure 4 shows a plot of xa(t) and x[n] for the case when fs=40
% samples/second. It was produced using the code below.
% 
%  include the necessary code 
fs = 40; 
ifig = 4; 


t = ct_dt(A,f0,PHI,fs,nc,ifig);
%%
% _Discussion of Figure 4_
%
% include discussion here.  
%
%%
% d) ...
%
%% Exercise 2
%
% 

close all
A = 6; 
f0 = 50; 
PHI = -pi/3; 
fs = 100; 
nc = 2;
ifig = 21; 


t = ct_dt(A,f0,PHI,fs,nc,ifig); 


A = 3; 
f0 = 50; 
PHI = 0; 
fs = 100; 
nc = 2;
ifig = 22; 


t = ct_dt(A,f0,PHI,fs,nc,ifig);


A = 6; 
f0 = 50; 
PHI = -pi/3; 
fs = 100; 
nc = 2;
ifig = 23; 


t = ct_dt(A,f0,PHI,fs,nc,ifig); 
y = 3*cos(2*pi*50*t);
plot(t,y,'*')
legend('x(t)','x[n]','y(t)')
%% Exercise 3
%
% 



close all
A = 1; 
f0 = 50; 
PHI = -pi/2; 
fs = 80; 
nc = 5;
ifig = 31; 


t = ct_dt(A,f0,PHI,fs,nc,ifig); 


y = cos(2*pi*30*t + pi/2);

plot(t,y,'*')
legend('x(t)','x[n]','y(t)')


%% Exercise 4
%
%
close all
A = 1; 
f0 = 60; 
PHI = 0; 
fs = 50; 
nc = 12;
ifig = 41; 


t = ct_dt(A,f0,PHI,fs,nc,ifig); 

y = cos(2*pi*10*t);

plot(t,y,'*')
legend('x(t)','x[n]','y(t)')

%% Appendix 1: Listing of the file ct_dt.m
%
%     Please include a listing of the function here, complete with any 
%     modifications that you may have made.
%
%     function t = ct_dt(A,f0,PHI,fs,nc, ifig)
%     %A 	    amplitude of cosine;
%     %f0 	CT frequency of cosine (cycles/sec);
%     %PHI 	phase of cosine (radians);
%     %fs	    sampling frequency (samples/sec.)
%     %nc 	number of CT cycles to be displayed
%     %ifig   optional Figure number to use in the title of the plot
%
%     ...

\end{lstlisting}
\subsection{The function of ct\_dt}
\begin{lstlisting}
function t = ct_dt(A,f0,PHI,fs,nc, ifig)
%A 	    amplitude of cosine;
%f0 	CT frequency of cosine (cycles/sec);
%PHI 	phase of cosine (radians);
%fs	    sampling frequency (samples/sec.)
%nc 	number of CT cycles to be displayed
%ifig   option Figure number to use in the title of the plot
%t      time vector used to plot CT cosine
if (nargin < 5 | nargin > 6)
    error('in call to ct_dt: there should be 5 arguments')
end
if (A < 0 )
    error(['in call to ct_dt: Amplitude of cosine, A,' ...
        'should not be negative'])
end
if (fs<0)
    error(['in call to ct_dt: the sampling frequency,'...
           ' fs,should be positive'])
end
if (nc<0)
    error('in call to ct_dt: nc should be positive')
end
if (exist('ifig'))
    pFig = ['Fig. ', num2str(ifig),':  '];
else
    pFig = [''];
end
    

figure, clf      
Ts=1/fs; %time between samples
Tp=1/abs(f0);  %period of CT cosine (sec/cycle)
F0 = f0/fs; %DT frequency (cycles/sample)
%DT plot will display samples n=0 to n=nmax
nmax = nc/abs(F0); 
%CT plot will display t=0 to t=tmax
tmax = nmax * Ts; 
% define t vector for CT plots to:
%  have a length greater than or equal to 200
%  with every kth element corresponding to a sampling instant
k = ceil(200/nmax);
t=0:Ts/k:tmax; 
xa = A*cos(2*pi*f0*t + PHI);
plot(t,xa);
hold on
n=0:nmax;
nTs = n*Ts;
xn = A*cos(2*pi*F0*n + PHI);
stem(nTs,xn);

p0=['x_a(t)=A cos(2\pi f_0 t), '];
p1=[' A='];
p2=[', f0='];
p3=[', PHI='];
p4=[', fs='];
p5=[', nc='];
p6=[', User='];
name=getenv('USER'); % gets the users login id
title( [pFig p0 p1 num2str(A) p2 num2str(f0) p3 num2str(PHI) ...
         p4 num2str(fs) p5 num2str(nc) p6 name ] )
xlabel('t (seconds), n')

\end{lstlisting}

\end{document}