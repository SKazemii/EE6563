\begin{frontmatter}

\title{\Huge EE6563 Project Proposal\\ Footprint Recognition based on the\\ Spatio-Temporal Features }


%% Group authors per affiliation:
\author{Saeed Kazemi\fnref{myfootnote}}
\address{University of New Brunswick}
\fntext[myfootnote]{Saeed.Kazemi@unb.ca.}

%% or include affiliations in footnotes:
%\ead[url]{https://github.com/SKazemii/EE6563}

\begin{abstract}
Given present-day security concerns, many buildings have implemented robust authentication techniques. Aside from authentication to enter a building, applications such as border and airport security also administer identification. Therefore, many cities and companies provide technologies like CCTV or fingerprinting for authentication and verification. But each system has its own drawbacks. For example, due to the Covid-19 pandemic, most people wear a mask and avoid touching unnecessary surfaces. In this research, we work on a new biometric system called gait recognition. This system has some benefits in comparison to CCTV and fingerprint. 
\end{abstract}

\begin{keyword}
Footprint recognition\sep Time series\sep pressure sensor
\end{keyword}

\end{frontmatter}
