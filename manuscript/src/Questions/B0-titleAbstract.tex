\begin{frontmatter}
%EE6563 Project Progress\\ 
\title{\Huge EE6563 - Final Report\\Footprint Recognition based on \\ Time-Series Classification }
%%$$\title{\Huge CS6735 Research Project \\Footprint Recognition based on \\ Time-series classification }

%% Group authors per affiliation:
\author{Saeed Kazemi\fnref{myfootnote}}
%%$$\author{Maryam Sepasi\fnref{myfootnote1}}
\address{University of New Brunswick}
\fntext[myfootnote]{Saeed.Kazemi@unb.ca.}
%%$$\fntext[myfootnote1]{maryam.sepasi@unb.ca.}

%% or include affiliations in footnotes:
%\ead[url]{https://github.com/SKazemii/EE6563}

\begin{abstract}
Given present-day security concerns, many buildings have implemented robust authentication techniques. Aside from authentication to enter a building, applications such as border and airport security also administer identification. Therefore, many cities and companies provide technologies like CCTV or fingerprinting for authentication and verification. But each system has its own drawbacks. For example, due to the Covid-19 pandemic, most people wear a mask and avoid touching unnecessary surfaces. Thus, gait recognition could be a solution. 



%In this project, we worked on temporal information of time-series data. These features used to construct a classifier. Moreover, verification mode was used for this research.

%This aids to disguise people and reduces the hygiene of fingerprint biometrics respectively. In this research, we focus on which classifier has better results on barefoot footprints and which features have the important effect on the classifier.


%%$$In this paper, we present a review of the time series approaches for classification tasks including conventional machine learning algorithms and Deep Neural Networks. Also, in \ref{appendix:2}, more approaches have been reviewed for time series classification. These approaches were implemented in verification mode. Experimental results show that the SVM classifier on the contribution of all handcrafted features had the best performance with 91.3\%.

In this paper, we present a review of the time series approaches for classification tasks including conventional machine learning algorithms and Deep Neural Networks. These approaches were implemented in verification mode. Experimental results show that the SVM classifier on the contribution of all handcrafted features had the best performance with 91.3\%.

%Convolutional Neural Networks (CNN) has achieved a great success in image recognition task by automaticallylearning a hierarchical feature representation from raw data. While the majority of Time-Series Classification(TSC) literature is focused on 1D signals, this paper uses Recurrence Plots (RP) to transform time-series into2D texture images and then take advantage of the deep CNN classifier.  Image representation of time-seriesintroduces different feature types that are not available for 1D signals, and therefore TSC can be treated astexture image recognition task. CNN model also allows learning different levels of representations together witha classifier, jointly and automatically. Therefore, using RP and CNN ina unified framework is expected to boostthe recognition rate of TSC. Experimental results on the UCR time-series classification archive demonstratecompetitive accuracy of the proposed approach, compared not only to the existing deep architectures, but alsoto the state-of-the art TSC algorithms.


\end{abstract}

\begin{keyword}
Footprint recognition\sep Time-series classification\sep pressure sensor
\end{keyword}

\end{frontmatter}
