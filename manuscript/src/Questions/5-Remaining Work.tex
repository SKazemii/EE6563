%\section{Constraints}
%The constraints in this project could fall into two groups. The first constraint is related to the limitation on laboratory conditions \cite{Connor2018BiometricFeatures}. In real-world circumstances, many situations such as walking speed, clothing, footwear, and load carriage could affect our results, whereas our datasets could cover some of these real-world conditions. For example, except for the walking speed and footwear, both datasets do not have useful information for other real-world situations. Consequently, our results are optimistically biased. Table \ref{tab:1_vul} indicates some inhibiting factors. 

%Another significant limitation in the Stepscan dataset is the lack of relative footprint location. The dataset included only aligned and segmented footprint images. The location of samples concerning each other is unknown. This information could play a significant role in predicting the location of future footsteps. 

%\begin{center}
%	\begin{table}[!t]
%	\caption{Some inhibiting factors in gait recognition.}
%	\label{tab:1_vul}
%	\hspace{3em}
%	\setlength\extrarowheight{-2pt}
%	\begin{tabular}{rl}
\toprule
    &  Inhibit Factors \\
\midrule
  1 & Footwear               \\
  2 & clothing               \\
  3 & Injury                 \\
  4 & Muscle development     \\
  5 & Fatigue                \\
  6 & Age                    \\
  7 & Load carriage          \\
  8 & Walking speed          \\
  
\bottomrule
\end{tabular}


%	\end{table}
%\end{center}
 

\section{Remaining Work}

In recent years, improvements in the computational process of computers and other benefits of \gls{dnn} have caused many researchers (like \cite{IsmailFawaz2019DeepReview} and \cite{Costilla-Reyes2018DeepSensors}) to move towards \gls{dnn} for the classification of time series data. Therefore, these algorithms will be reviewed in a later stage of this project.

Because the entire time series were considered for extracting features, we have had one value for each attribute. For instance, while local maximums could be good features, the algorithm returned only the global maximum as a Max feature. We could slide a window on the signal to extract handcrafted features. It causes only a part of the data to consider in each moment. In other words, because each data includes several stages of the walking cycle, features should be extracted according to these periods.   

Moreover, cropping the heel or toe area in each video could separate features based on the walking stages (pressure area). By this means, each video will be divided into N sub-area, and then for each region, we reconstruct time-series signals and extract features. As a result, these features belong only to that area or stage of walking.

%Due to using the aligned video, we could consider some pixels over time as 2D signals. Although our features space increased rapidly, we could put a threshold on variance and energy of time-series signals for controlling features space. 

In this research, regardless of eliminating high-correlated and low variance features, we should use a more complex feature selection algorithm to eliminated irrelevant and redundant ones. This reduction might increase the accuracy.

%In addition, it had better to split the dataset into left and right footprints for exploring the effects of extracted features on classifier accuracy. 
 


